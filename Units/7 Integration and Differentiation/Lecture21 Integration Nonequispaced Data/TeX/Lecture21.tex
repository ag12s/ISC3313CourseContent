\documentclass{if-beamer}

% --------------------------------------------------- %
%                  Presentation info	              %
% --------------------------------------------------- %
\title[Lecture 21]{Lecture 21}
\subtitle{Integration with Unequal Segments}
\author{Ashley Gannon}
\date{ISC3313 Fall 2021}
\logo{
\includegraphics[scale=0.08]{figures/FSULogo.png}
}
\subject{Presentation subject}

% --------------------------------------------------- %
%                    Title + Schedule                 %
% --------------------------------------------------- %
\begin{document}

\begin{frame}
  \titlepage
\end{frame}
% --------------------------------------------------- %
%                      Presentation                   %
% --------------------------------------------------- %
\section{Integration with Unequal Segments}
\begin{frame}
\frametitle{Integration with Unequal Segments}
To this point, all formulas for numerical integration have been based on equispaced data.\\\vspace{10pt}

In practice, there are many situations where this assumption does not hold and we must deal with unequal-sized segments. For example, experimentally derived data are often of this type. \\\vspace{10pt}

For these cases, we need to change the way we define $h$. \\\vspace{10pt}

Let's write a Trapezoid Rule function that takes in non equispaced data.
\end{frame}

\begin{frame}
\frametitle{Thinking about this in regards to the Trapezoid Rule}
Recall the approximate value for the integral with equispaced data was
$$I = \frac{h}{2}(f(x_0)+f(x_n))+h\left( \sum_{i = 1}^{n-1}f(x_i)\right)$$
Since our h value isn't the same for each segment, we can no longer write it this way. Instead, 
$$I = \frac{h_1}{2}(f(x_0)+f(x_1)) + \frac{h_2}{2}(f(x_1)+f(x_2))
+...+ \frac{h_n}{2}(f(x_{n-1})+f(x_n))$$

Where $h_1$ is the width of the first segment, $h_2$ is the width of the second segment and so on. So, instead of defining $h$ once outside out loop, we need to calculate it each time we iterate. 
\end{frame}

\begin{frame}
	\frametitle{The Composite Trapezoid Rule Pseudocode}
		
	\begin{minipage}[t]{0.5\textwidth}
	Our old pseudocode:\\
	\vspace{10pt}
	\texttt{function takes in a, b, n}\\
	\texttt{declare/define h = (b-a)/n;}\\
	\texttt{declare/define sum = (h/2)*(f(a)+f(b)); }\\
	\texttt{declare/define xi = a+h;}\\
	\texttt{ }\\
	\texttt{loop over i = 1,...,n-1 }\\
	\texttt{\qquad add h*f(xi) to sum} \\
	\texttt{\qquad update xi = xi +h;}\\
	\end{minipage}
	\vspace{10pt}
	\begin{minipage}[t]{0.5\textwidth}
	Pseudocode for non equispaced data\\
	\vspace{10pt}
	\texttt{function takes in x, fx, n}\\
	\texttt{declare/define h = 0;}\\
	\texttt{declare/define sum = 0; }\\
	\texttt{ }\\
	\texttt{loop over i = 1,...,n }\\
	\texttt{\qquad calculate new h = x[i]-x[i-1]}\\
	\texttt{\qquad add (h/2)*(fx[i-1]+fx[i]) to sum} \\
	\end{minipage}	
	

	
\end{frame}	

\section{Importing Data}

\begin{frame}
	\frametitle{Reading in data}
	In order to write and read files we need to include header file
	\texttt{<fstream>}. \\\vspace{5pt}
	Reading from file can be done by declaring \texttt{ifstream} in the following manner:\\\vspace{10pt}
	
	\texttt{// create a binary file object and open a file that contains the data}\\
	\texttt{ifstream xdata("xdata.dat", ifstream::binary);;}\\
	\texttt{ }\\
	\texttt{// write data to file}\\
	\texttt{for (int i = 0; i < N; i++)}\\
	\texttt{\{}\\
	\texttt{\qquad // this takes the data on one line and stores in x[i]}\\
	\texttt{\qquad xdata >> x[i];}\\
	\texttt{\}}\\
	\texttt{ }\\
	\texttt{// close file}\\
	\texttt{xdata.close();}\\
	
\end{frame}

\begin{frame}
	\frametitle{Reading in data}
	In order to write and read files we need to include header file
	\texttt{<fstream>}. \\\vspace{5pt}
	Reading from file can be done by declaring \texttt{ifstream} in the following manner:\\\vspace{10pt}
	
	\texttt{// create a binary file object and open a file that contains the data}\\
	\texttt{ifstream xdata("xdata.dat", ifstream::binary);;}\\
	\texttt{ }\\
	\texttt{// write data to file}\\
	\texttt{for (int i = 0; i < N; i++)}\\
	\texttt{\{}\\
	\texttt{\qquad // this takes the data on one line and stores in x[i]}\\
	\texttt{\qquad xdata >> x[i];}\\
	\texttt{\}}\\
	\texttt{ }\\
	\texttt{// close file}\\
	\texttt{xdata.close();}\\
	\vspace{10pt}
	But what if we don't know how many elements our data has?
\end{frame}

\begin{frame}
\frametitle{C++ Vectors}
In order to use vectors, we need to include header file
\texttt{<vector>}. \\\vspace{10pt}
Vectors are sequence containers representing arrays that can change in size.\\\vspace{10pt}

Internally, vectors use a dynamically allocated array to store their elements. \\\vspace{5pt}

They are declared as follows:
\\\vspace{5pt}
\texttt{vector<double> x}

\end{frame}

\begin{frame}
	\frametitle{C++ Vectors}
	In order to use vectors, we need to include header file
	\texttt{<vector>}. \\\vspace{10pt}
	Vectors are sequence containers representing arrays that can change in size.\\\vspace{10pt}
	
	Internally, vectors use a dynamically allocated array to store their elements. \\\vspace{10pt}
	
	They are declared as follows:
	\\\vspace{10pt}
	\texttt{vector<double> X}\\
	\vspace{10pt}
	If we want to save each element of the data file to the vector, we can use the \texttt{push\_back} member function. This member function adds a new element to the end of the vector, \texttt{x}.
	\\\vspace{10pt}
	\texttt{double x}\\
	\texttt{while (xdata >> x) }\\
	\texttt{\qquad X.push\_back(x);}\\
	\texttt{xdata.close()}\\
\end{frame}

\begin{frame}
	\frametitle{Reading in x and fx}
	Download "fdata.dat" and "xdata.dat" from Canvas, you'll need them for this. \\\vspace{10pt}
%	
	\texttt{ifstream xdata("xdata.dat", ifstream::binary)} \\ 
	\texttt{ifstream fdata("fdata.dat", ifstream::binary)} \\
	\texttt{ }\\
	\texttt{vector<double> X,Fx;}
	\texttt{ }\\
	\texttt{double x;}\\
	\texttt{while (xdata >> x)}\\	
	\texttt{\qquad X.push\_back(x)}\\
	\texttt{xdata.close()}\\	
	\texttt{ }\\
	\texttt{double fx;}\\
	\texttt{while (fdata >> fx)}\\	
	\texttt{\qquad Fx.push\_back(fx)}\\	
	\texttt{fdata.close()}\\\vspace{10pt}
	
	To get the size of the vector, we use the public member function \texttt{size}. \\\vspace{10pt}
	\texttt{int n = X.size() - 1}\\	
\end{frame}

\end{document}

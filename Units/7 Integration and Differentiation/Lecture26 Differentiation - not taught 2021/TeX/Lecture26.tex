\documentclass{if-beamer}

% --------------------------------------------------- %
%                  Presentation info	              %
% --------------------------------------------------- %
\title[Lecture 26]{Differentiation}
\subtitle{Richardson Extrapolation}
\author{Ashley Gannon}
\date{Fall 2020}
\logo{
\includegraphics[scale=0.08]{figures/FSULogo.png}
}
\subject{Presentation subject}

% --------------------------------------------------- %
%                    Title + Schedule                 %
% --------------------------------------------------- %
\begin{document}

\begin{frame}
  \titlepage
\end{frame}
% --------------------------------------------------- %
%                      Presentation                   %
% --------------------------------------------------- %
\section{Richardson Extrapolation}

\begin{frame}
	\frametitle{Richardson Extrapolation Revisited}
	
	So far, we've seen that we can increase the accuracy of our derivative approximation by decreasing the step size $h$.\\\vspace{1cm}
	
	We could also improve these estimates by using higher order taylor series to form our schemes\\\vspace{1cm}
	
	We're going to use Richardson Extrapolation. Like we did for integration, this method combines two derivative estimates to compute a third, more accurate approximation.
	
\end{frame}


\begin{frame}
	\frametitle{Richardson Extrapolation}
	Recall from the unit on integration
	$$I = \frac{4}{3}I(h_2)-\frac{1}{3}I(h_1)$$
	where $I(h_1)$ and $I(h_2)$ are the integral estimates for step sizes $h_1$ and $h_2$. \\\vspace{1cm}
	We can write a similar expression for derivatives
	$$ D = \frac{4}{3}D(h_2)-\frac{1}{3}D(h_1) $$
	where $D(h_1)$ and $D(h_2)$ are the derivative estimates for step sizes $h_1$ and $h_2$.
	\begin{itemize}
		\item For Forward or Backward difference approximations with $\mathcal{O}(h)$, the application of this formula will yield a new derivative estimate of $\mathcal{O}(h^2)$
		\item For Centered difference approximations with $\mathcal{O}(h^2)$, the application of this formula will yield a new derivative estimate of $\mathcal{O}(h^4)$
	\end{itemize}
	
\end{frame}

\begin{frame}
	\frametitle{Richardson Extrapolation - Forward Difference}
	Use forward difference approximations to estimate the first derivative of
	$$f(x) = -0.1x^4-0.15x^3-0.5x^2-0.25x+1.2 $$
	at $x=0.5$, using a step size $h = 0.25$. Note that the derivative of this function is
	$$f'(x) = -0.4x^3-0.45x^2-x-0.25, \qquad f'(0.5) = -0.9125$$
	\\\vspace{5pt}
	Starting again with the forward difference scheme.\\\vspace{5pt}
	We have
	when $h=0.5$,
	$$f'(0.5) \approx \frac{0.2-0.925}{0.5} = -1.45$$
	The percent relative error is $58.9\%$.
	When $h = 0.25$,
	$$f'(0.5) \approx \frac{0.63632813-0.925}{0.5} \approx -1.155 $$
	The percent relative error is $26.5\%$.
\end{frame}

\begin{frame}
	\frametitle{Richardson Extrapolation - Forward Difference}
	Plugging these values into out Richardson Extrapolation scheme
	$$D = \frac{4}{3}D(h_2)-\frac{1}{3}D(h_1) $$
	We have
	$$D = \frac{4}{3}(-1.155)-\frac{1}{3}(-1.45) = -1.05667$$
	Which has a percent relative error of $15.8\%$, which is close to the error from the centered difference approximation.
\end{frame}

\begin{frame}
	\frametitle{Richardson Extrapolation - Centered Difference}
	Use a centered difference approximation to estimate the first derivative of
	$$f(x) = -0.1x^4-0.15x^3-0.5x^2-0.25x+1.2 $$
	at $x=0.5$. Note that the derivative of this function is
	$$f'(x) = -0.4x^3-0.45x^2-x-0.25, \qquad f'(0.5) = -0.9125$$
	\\\vspace{5pt}
	When $h=0.5$, applying the centered difference scheme we have
	$$f'(0.5) \approx \frac{0.2-1.2}{1} = -1 $$
	The percent relative error is $9.6\%$.
	When $h=0.25$, applying the centered difference scheme we have
	$$f'(0.5) \approx \frac{0.63632813-1.10351563}{0.5} = -0.934$$
	The percent relative error is $2.4\%$
\end{frame}

\begin{frame}
	\frametitle{Richardson Extrapolation - Centered Difference}
	Plugging these values into out Richardson Extrapolation scheme
	$$D = \frac{4}{3}D(h_2)-\frac{1}{3}D(h_1) $$
	We have
	$$D = \frac{4}{3}(-0.934)-\frac{1}{3}(-1) = -0.9125$$
	which for the present case is exact.
	\begin{itemize}
		\item Recall the Richardson extrapolation is equivalent to fitting a high-order polynomial through data, in this case it's 4th order so it fits our 4th order polynomial $f(x)$ exactly.
	\end{itemize}
\end{frame}

\section{Wrapping Up}
\begin{frame}
	\frametitle{Differentiation wrapping up}
	This section covered one-step methods for solving ordinary differential equations (ODEs).\\\vspace{1cm} 
	
	If you continue taking courses in Scientific Computing, you will apply these methods and some of the other methods we've covered in this course to solve ODEs. 
\end{frame}

\end{document}

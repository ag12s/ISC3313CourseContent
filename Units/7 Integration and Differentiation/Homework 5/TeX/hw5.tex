\documentclass[11pt]{article}
\usepackage{amsmath,amssymb,amsthm}
\usepackage{tabu}
\usepackage{graphicx}
%\usepackage[bw]{mcode}
\usepackage[margin=.6in]{geometry}
\usepackage{tikz}
\usepackage{float}
\usepackage{textcomp}
\usepackage{multicol}
\addtolength{\topmargin}{.5in}
\usepackage{fancyhdr}
\usetikzlibrary{positioning}
\usepackage{pgfplots}
\usepackage{enumitem}
\setlength{\parindent}{0pt}
\setlength{\parskip}{5pt plus 1pt}
\setlength{\headheight}{20pt}
\renewcommand{\headrulewidth}{0pt}
\setlength{\headheight}{30.0pt}
\newcommand\question[2]{\vspace{.25in}\hrule\textbf{#1: #2}\vspace{.5em}\hrule\vspace{.10in}}
\renewcommand\part[1]{\vspace{.10in}\textbf{(#1)}}
\newcommand\enter{\vspace{.50in}}
\newcommand\algorithm{\vspace{.10in}\textbf{Algorithm:}}
\newcommand\correctness{\vspace{.10in}\textbf{Correctness: }}
\newcommand\runtime{\vspace{.10in}\textbf{Running time: }}
\pagestyle{fancyplain}

\begin{document}\raggedright
\newcommand\Page{\page  / \lastPage}
\newcommand\page{1}
\newcommand\qN[2]{\Large {#1} \small{#2} \normalsize}

% info
\newcommand\dueDate{\today}
\newcommand\hwnum{5}
\newcommand\ExNum{}

\newcommand\lastPage{3}

% set info
\lhead{\large Homework \hwnum }
\rhead{\rightHead}
\chead{\LARGE{Optimization, Integration, and Library Upkeep}}
\newcommand\rightHead{\large Due Nov 27, 2021}

\question{1}{The golden-section search and parabolic interpolation }
Consider the following function:
$$f(x) = x^4 + 2x^3 +8x^2+5x $$
\begin{enumerate}[label = (\alph*)]
	\item Use an analytical and/or graphical method to show the function has a minimum for some value of $x$ in the domain $-2\leq x\leq1$.
	\item Use the golden-section search with $x_l = -2$, $x_u = 1$, tol = 1\%. \textbf{Report} the minimum.
	\item Use parabolic interpolation with $x_1 = -2$, $x_2 = -1$, and $x_3 = 1$ for 5 iterations. \textbf{Report} the minimum.
\end{enumerate}
\vspace{4.5cm}
\question{2}{The composite trapezoid rule}
Add the Composite Trapezoid Rule to your library in a new class "Integration". Use it to determine the distance fallen by the free-falling bungee jumper in the first 3 seconds by evaluating the integral of
	$$v(t) = \sqrt{\frac{gm}{c_d}}tanh\left(\sqrt{\frac{gc_d}{m}}t\right). $$
The second derivative of $v(t)$ is 
	$$v''(t) = -2g^{3/2}\sqrt{\frac{c}{m}}tanh\left(t\sqrt{\frac{cg}{m}}\right)sech^2\left(t\sqrt{\frac{cg}{m}}\right)$$ 
	which is used to find the average of the second derivative
	$$v''(t) = \frac{\int_{0}^{3}-2g^{3/2}\sqrt{\frac{c}{m}}tanh\left(t\sqrt{\frac{cg}{m}}\right)sech^2\left(t\sqrt{\frac{cg}{m}}\right)}{3-0} = -0.8668$$ 
	Apply the recursive composite trapezoid rule to this problem with a tolerance of 0.00001, and the following parameter values: $g = 9.81$ m/s2, $m = 68.1$ kg, and
	$c_d = 0.25$ kg/m. \textbf{Report} 
	\begin{enumerate}
		\item the distance fallen by the free-falling bungee jumper
		\item the number of segments needed to reach the tolerance
	\end{enumerate}  


\newpage
\question{3}{The composite Simpson's 1/3 rule}
Add the Composite Simpson's 1/3 Rule to your library in the appropriate class. Use it to determine the distance fallen by the free-falling bungee jumper in the first 3 seconds by evaluating the integral of
$$v(t) = \sqrt{\frac{gm}{c_d}}tanh\left(\sqrt{\frac{gc_d}{m}}t\right). $$
The fourth derivative of $v(t)$ is 
$$v^{(4)}(t) =\sqrt{\frac{gm}{c_d}}\left(\frac{16c_d^2g^2tanh\left(t\sqrt{\frac{c_dg}{m}}\right)sech^4\left(t\sqrt{\frac{c_dg}{m}}\right)}{m^2} -\frac{8c_d^2g^2tanh^3\left(t\sqrt{\frac{c_dg}{m}}\right)sech^2\left(t\sqrt{\frac{c_dg}{m}}\right)}{m^2} \right)$$
which is used to find the average of the fourth derivative
$$v^{(4)}(t) = \frac{\int_{0}^{3}\sqrt{\frac{gm}{c_d}}\left(\frac{16c_d^2g^2tanh\left(t\sqrt{\frac{c_dg}{m}}\right)sech^4\left(t\sqrt{\frac{c_dg}{m}}\right)}{m^2} -\frac{8c_d^2g^2tanh^3\left(t\sqrt{\frac{c_dg}{m}}\right)sech^2\left(t\sqrt{\frac{c_dg}{m}}\right)}{m^2} \right)}{3-0} = 0.2001$$ 
NOTE: The average value for the 4th derivative is a positive value so you will have to change the if statement in your code to \texttt{if (abs(error)>tol)} to for your recursion code to work since $E_a < 0$. You should change this in both methods for your code to be robust.\\\vspace{0.5cm}
Apply the recursive composite Simpson's 1/3 rule to this problem with a tolerance of 0.00001, and the following parameter values: $g = 9.81$ m/s2, $m = 68.1$ kg, and
$c_d = 0.25$ kg/m. \textbf{Report:} 
\begin{enumerate}
	\item the distance fallen by the free-falling bungee jumper
	\item the number of segments needed to reach the tolerance
\end{enumerate}
\vspace{0.5cm}

\question{4}{Integrating a data set }
You've just been handed a set of experimental data to integrate, \texttt{xData} and \texttt{fxData}. 
\begin{enumerate}[label = (\alph*)]
	\item Of the methods we've covered during lecture for integrating sets of data, which method would you choose use to integrate this data set? Why? \vspace{2cm}
	\item Apply the method you chose to the set of data provided with this assignment. Report the integral value (sum).
\end{enumerate}

\newpage
\question{5}{Submit Your Library  }
By this point in time your library should contain the following routines:
\begin{itemize}
	\item The Bisection method, \\\texttt{Bisect(double a, double b, double tol, double (*f)(double x))}
	\item Newton-Raphson method, \\\texttt{NewtonRaphson(double a, double tol, double maxit, double (*f)(double x), double (*df)(double x))}
	\item Golden-Section Search Method, \\\texttt{GoldenSectionSearch(double xu, double xl, double tol, double (*f)(double x))}
	\item Parabolic Interpolation, \\\texttt{ParabolicInterpolation(double x1, double x2, double x3, double tol, double (*f)(double x))}
	\item The Recursive Trapezoid Rule for functions, \\\texttt{CompositeTrapezoid(double a, double b, int n, double avgdf2, double tol, double (*f)(double x))}
	\item Recursive Simpson's 1/3 Rule for functions, \\\texttt{CompositeSimpsons13(double a, double b, int n, double avgdf4, double tol, double (*f)(double x))}
	\item Simpson's Rules for data, \\\texttt{DataSimpsons(double data[], double x[], int n)}
	\item Trapezoid Rule for data, \\\texttt{DataTrapezoid(vector<double> x, vector<double> fx, int n)}
\end{itemize}
Make sure that these routines work by testing them on the functions we covered in class, if they are not working you will not receive full credit for them. You are not required to submit any output for these tests, but we will check them to make sure they are working. 

\question{6}{Submission Details}
\textbf{Only} the functions used for Questions 1 and 3 of this assignment should be defined within your Main.cpp.\\\vspace{5pt}
Within the main function of your Main.cpp, you should be \textbf{only} be calling
\begin{enumerate}
	\item \texttt{GoldenSectionSearch}
	\item \texttt{ParabolicInterpolation}
	\item \texttt{CompositeTrapezoid}
	\item \texttt{CompositeSimpsons13}
	\item \texttt{DataSimpsons}
\end{enumerate}
with their appropriate parameters and reporting their returned values. Submit your entire SciProgLib folder as a .zip file.
\end{document}

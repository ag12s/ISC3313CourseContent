\documentclass{if-beamer}

% --------------------------------------------------- %
%                  Presentation info	              %
% --------------------------------------------------- %
\title[Lecture 24]{Lecture 24}
\subtitle{Numerical Differentiation}
\author{Ashley Gannon}
\date{ISC3313 Fall 2021}
\logo{
\includegraphics[scale=0.08]{figures/FSULogo.png}
}
\subject{Presentation subject}

% --------------------------------------------------- %
%                    Title + Schedule                 %
% --------------------------------------------------- %
\begin{document}

\begin{frame}
  \titlepage
\end{frame}
% --------------------------------------------------- %
%                      Presentation                   %
% --------------------------------------------------- %
\section{Introduction}

\begin{frame}[t]
\frametitle{Introduction}
Recall that the velocity of a free-falling bungee jumper as a function of time can be
formulated as
$$ v(t) = \sqrt{\frac{gm}{c_d}}tanh\left(\sqrt{\frac{gc_d}{m}t}\right) $$
\end{frame}

\begin{frame}[t]
	\frametitle{Introduction}
	Recall that the velocity of a free-falling bungee jumper as a function of time can be
	formulated as
	$$ v(t) = \sqrt{\frac{gm}{c_d}}tanh\left(\sqrt{\frac{gc_d}{m}t}\right) $$
	At the beginning of the integration unit, we used calculus to integrate this equation to determine the
	vertical distance $y$ the jumper has fallen after a time $t$.
	$$y(t) = \frac{m}{c_d}\ln\left[cosh\left( \sqrt{\frac{gc_d}{m}t} \right) \right]$$
\end{frame}

\begin{frame}[t]
	\frametitle{Introduction}
	Recall that the velocity of a free-falling bungee jumper as a function of time can be
	formulated as
	$$ v(t) = \sqrt{\frac{gm}{c_d}}tanh\left(\sqrt{\frac{gc_d}{m}t}\right) $$
	At the beginning of the integration unit, we used calculus to integrate this equation to determine the
	vertical distance $y$ the jumper has fallen after a time $t$.
	$$y(t) = \frac{m}{c_d}\ln\left[cosh\left( \sqrt{\frac{gc_d}{m}t} \right) \right]$$
	Now suppose that you were given the reverse problem. That is, you were asked to
	determine velocity based on the jumper’s position as a function of time. Because it is the inverse of integration, differentiation could be used to make the determination:
	$$v(t) = \frac{dz(t)}{dt} = \sqrt{\frac{gm}{c_d}}tanh\left(\sqrt{\frac{gc_d}{m}t}\right) $$
\end{frame}

\begin{frame}[t]
	\frametitle{Introduction}
	Recall that the velocity of a free-falling bungee jumper as a function of time can be
	formulated as
	$$ v(t) = \sqrt{\frac{gm}{c_d}}tanh\left(\sqrt{\frac{gc_d}{m}t}\right) $$
	At the beginning of the integration unit, we used calculus to integrate this equation to determine the
	vertical distance $y$ the jumper has fallen after a time $t$.
	$$y(t) = \frac{m}{c_d}\ln\left[cosh\left( \sqrt{\frac{gc_d}{m}t} \right) \right]$$
	Now suppose that you were given the reverse problem. That is, you were asked to
	determine velocity based on the jumper’s position as a function of time. Because it is the inverse of integration, differentiation could be used to make the determination:
	$$v(t) = \frac{dy(t)}{dt} = \sqrt{\frac{gm}{c_d}}tanh\left(\sqrt{\frac{gc_d}{m}t}\right) $$
	
	Beyond velocity, you might also be asked to compute the jumper’s acceleration. To
	do this, we could either take the first derivative of velocity, or the second derivative of
	displacement:
	$$a(t) = \frac{dv(t)}{dt} = \frac{d^2y(t)}{dt^2} = gsech^2\left( \sqrt{\frac{gc_d}{m}t} \right)$$
	
	\end{frame}

\begin{frame}[t]
	\frametitle{Introduction}
	\begin{itemize}
		\item Although a closed-form solution can be developed for this case, there are other functions that may be difficult or impossible to differentiate analytically. \vspace{1cm}
	\end{itemize}
\end{frame}

\begin{frame}[t]
	\frametitle{Introduction}
	\begin{itemize}
		\item Although a closed-form solution can be developed for this case, there are other functions that may be difficult or impossible to differentiate analytically. \vspace{1cm}
		\item Because engineers and scientists must continuously deal with systems and processes that change, calculus is an essential tool of our
		profession. Standing at the heart of calculus is the mathematical concept of differentiation. \vspace{1cm}
	\end{itemize}
\end{frame}

\begin{frame}[t]
	\frametitle{Introduction}
	\begin{itemize}
		\item Although a closed-form solution can be developed for this case, there are other functions that may be difficult or impossible to differentiate analytically. \vspace{1cm}
		\item Because engineers and scientists must continuously deal with systems and processes that change, calculus is an essential tool of our
		profession. Standing at the heart of calculus is the mathematical concept of differentiation. \vspace{1cm}
		\item Mathematically, the derivative, which serves as the fundamental vehicle for differentiation, represents the rate of change
		of a dependent variable with respect to an independent variable. \vspace{1cm}
	\end{itemize}
\end{frame}
\section{Differentiation}
\begin{frame}[t]
	\frametitle{Differentiation}
	\begin{figure}
		\centering
		\includegraphics[width = 0.7\textwidth]{figures/figure1}
	\end{figure}
	\begin{itemize}
		\item As depicted in figure (a) above, the
		mathematical definition of the derivative begins with a difference approximation:
		$$\frac{\Delta y}{\Delta x} = \frac{f(x_i+\Delta x)- f(x_i)}{\Delta x} $$
		where $y$ and $f(x)$ are alternative representatives for the dependent variable and $x$ is the independent variable. 
		\end{itemize}	
\end{frame}

\begin{frame}[t]
	\frametitle{Differentiation}
	\begin{figure}
		\centering
		\includegraphics[width = 0.7\textwidth]{figures/figure1}
	\end{figure}
	\begin{itemize}
		\item As depicted in figure (a) above, the
		mathematical definition of the derivative begins with a difference approximation:
		$$\frac{\Delta y}{\Delta x} = \frac{f(x_i+\Delta x)- f(x_i)}{\Delta x} $$
		where $y$ and $f(x)$ are alternative representatives for the dependent variable and $x$ is the independent variable. 
		\item If $\Delta x$ is allowed to approach 0, demonstrated in figures (a) to (c), the difference becomes the derivative
		$$\frac{dy}{dx} = \lim_{\Delta x \rightarrow 0} \frac{f(x_i+\Delta x)- f(x_i)}{\Delta x}$$
	\end{itemize}	
\end{frame}

\begin{frame}[t]
	\frametitle{Differentiation}
	\begin{figure}
		\centering
		\includegraphics[width = 0.7\textwidth]{figures/figure1}
	\end{figure}
	\begin{itemize}
		\item As depicted in figure (a) above, the
		mathematical definition of the derivative begins with a difference approximation:
		$$\frac{\Delta y}{\Delta x} = \frac{f(x_i+\Delta x)- f(x_i)}{\Delta x} $$
		where $y$ and $f(x)$ are alternative representatives for the dependent variable and $x$ is the independent variable. 
		\item If $\Delta x$ is allowed to approach 0, demonstrated in figures (a) to (c), the difference becomes the derivative
		$$\frac{dy}{dx} = \lim_{\Delta x \rightarrow 0} \frac{f(x_i+\Delta x)- f(x_i)}{\Delta x}$$
		\item The derivative is the slope of the tangent line to to curve at $x_i$
	\end{itemize}	
\end{frame}
\section{Taylor Series}
\begin{frame}[t]
	\frametitle{Taylor Series}
	\begin{itemize}
		\item Taylor’s theorem and its associated formula, the Taylor series, is the foundation of some numerical methods. 
	\end{itemize}	
\end{frame}

\begin{frame}[t]
	\frametitle{Taylor Series}
	\begin{itemize}
		\item Taylor’s theorem and its associated formula, the Taylor series, is the foundation of some numerical methods. 
		\item In essence, the Taylor theorem states that any smooth function can
		be approximated as a polynomial.
	\end{itemize}	
\end{frame}

\begin{frame}[t]
	\frametitle{Taylor Series}
	\begin{itemize}
		\item Taylor’s theorem and its associated formula, the Taylor series, is the foundation of some numerical methods. 
		\item In essence, the Taylor theorem states that any smooth function can
		be approximated as a polynomial.
		\item The Taylor series then provides a means to express this
		idea mathematically in a form that can be used to generate practical results.
		\begin{figure}
			\centering
			\includegraphics[width = .6\textwidth]{figures/figure2}
		\end{figure}
	\end{itemize}	
\end{frame}

\section{Deriving the Forward and Backward Difference Schemes}

\begin{frame}[t]
	\frametitle{Taylor Series}
	
	If we expand the Taylor series forward, we have that 
	\begin{itemize}
		\item 	The zero-order Taylor-series approximation is
		$$f(x_{i+1}) \cong f(x_i) $$
		
	\end{itemize}	
\end{frame}


\begin{frame}[t]
	\frametitle{Taylor Series}
	
	If we expand the Taylor series forward, we have that 
	\begin{itemize}
		\item 	The zero-order Taylor-series approximation is
		$$f(x_{i+1}) \cong f(x_i) $$
		\item The first-order Taylor-series approximation is
		$$f(x_{i+1}) \cong f(x_i) +\frac{h}{1!}f'(x_i) $$
	\end{itemize}	
\end{frame}


\begin{frame}[t]
	\frametitle{Taylor Series}
	
	If we expand the Taylor series forward, we have that 
	\begin{itemize}
		\item 	The zero-order Taylor-series approximation is
		$$f(x_{i+1}) \cong f(x_i) $$
		\item The first-order Taylor-series approximation is
		$$f(x_{i+1}) \cong f(x_i) +\frac{h}{1!}f'(x_i) $$
		\item The second-order Taylor series approximation is 
		$$f(x_{i+1}) \cong f(x_i) +\frac{h}{1!}f'(x_i) + \frac{h^2}{2!}f''(x_i)$$
	\end{itemize}	
\end{frame}


\begin{frame}[t]
	\frametitle{Taylor Series}
	
	If we expand the Taylor series forward, we have that 
	\begin{itemize}
		\item 	The zero-order Taylor-series approximation is
		$$f(x_{i+1}) \cong f(x_i) $$
		\item The first-order Taylor-series approximation is
		$$f(x_{i+1}) \cong f(x_i) +\frac{h}{1!}f'(x_i) $$
		\item The second-order Taylor series approximation is 
		$$f(x_{i+1}) \cong f(x_i) +\frac{h}{1!}f'(x_i) + \frac{h^2}{2!}f''(x_i)$$
		\item If we continue this trend, the $n^{th}$-order Taylor series approximate can be written as
		$$f(x_{i+1}) \cong f(x_i) + \frac{h}{1!}f'(x_i) + \frac{h^2}{2!}f''(x_i)+\frac{h^3}{3!}f^{(3)}(x_i)+...+\frac{h^n}{n!}f^{(n)}(x_i)$$
	\end{itemize}	
\end{frame}

%%%%%%%%%%%%%%%%%%%%%%%%%%%%%%%%%%%%%%%%%%%%%%%%%%%%%%%%%%%%%%%%%%%%%%%

\begin{frame}[t]
	\frametitle{Approximating Derivatives - Forward finite difference}
	Taylor series can be used to approximate derivatives. If we wish to approximate the first derivative, we rearrange the first-order Taylor series approximation
	$$f(x_{i+1}) \cong f(x_i)+ \frac{h}{1!}f'(x_i),$$
	to solve for the derivative, $f'(x_i)$
	$$f'(x_i) = \frac{f(x_{i+1})-f(x_i)}{h} $$
\end{frame}

\begin{frame}[t]
	\frametitle{Approximating Derivatives - Forward finite difference}
	Taylor series can be used to approximate derivatives. If we wish to approximate the first derivative, we rearrange the first-order Taylor series approximation
	$$f(x_{i+1}) \cong f(x_i)+ \frac{h}{1!}f'(x_i),$$
	to solve for the derivative, $f'(x_i)$
	$$f'(x_i) = \frac{f(x_{i+1})-f(x_i)}{h} $$
	If we have equispaced data, we can think of $x_{i+1}$ as being the x located distance $h$ away from $x_i$. We can rewrite out formula as
	$$f'(x) = \frac{f(x+h)-f(x)}{h}  $$
	
\end{frame}

\begin{frame}[t]
	\frametitle{Approximating Derivatives - Forward finite difference}
	Taylor series can be used to approximate derivatives. If we wish to approximate the first derivative, we rearrange the first-order Taylor series approximation
	$$f(x_{i+1}) \cong f(x_i)+ \frac{h}{1!}f'(x_i),$$
	to solve for the derivative, $f'(x_i)$
	$$f'(x_i) = \frac{f(x_{i+1})-f(x_i)}{h} $$
	If we have equispaced data, we can think of $x_{i+1}$ as being the x located distance $h$ away from $x_i$. We can rewrite out formula as
	$$f'(x) = \frac{f(x+h)-f(x)}{h}  $$
	\\\vspace{10pt}
	This is the \textit{forward difference} scheme and it's approximation is $\mathcal{O}(h)$ accurate.
\end{frame}


%%%%%%%%%%%%%%%%%%%%%%%%%%%%%%%%%%%%%%%%%%%%%%%%%%%%%%%%%%%%%%%%%%%%%%%

\begin{frame}[t]
	\frametitle{Taylor Series}
	
	If we expand the Taylor series backward, we have that 
	\begin{itemize}
		\item 	The zero-order Taylor-series approximation is
		$$f(x_{i-1}) \cong f(x_i) $$
	\end{itemize}	
	
\end{frame}

\begin{frame}[t]
	\frametitle{Taylor Series}
	
	If we expand the Taylor series backward, we have that 
	\begin{itemize}
		\item 	The zero-order Taylor-series approximation is
		$$f(x_{i-1}) \cong f(x_i) $$
		\item The first-order Taylor-series approximation is
		$$f(x_{i-1}) \cong f(x_i) -\frac{h}{1!}f'(x_i) $$
	\end{itemize}	
	
\end{frame}

\begin{frame}[t]
	\frametitle{Taylor Series}
	
	If we expand the Taylor series backward, we have that 
	\begin{itemize}
		\item 	The zero-order Taylor-series approximation is
		$$f(x_{i-1}) \cong f(x_i) $$
		\item The first-order Taylor-series approximation is
		$$f(x_{i-1}) \cong f(x_i) -\frac{h}{1!}f'(x_i) $$
		\item The second-order Taylor series approximation is 
		$$f(x_{i-1}) \cong f(x_i) -\frac{h}{1!}f'(x_i) + \frac{h^2}{2!}f''(x_i)$$
	\end{itemize}	
	
\end{frame}

\begin{frame}[t]
	\frametitle{Taylor Series}
	
	If we expand the Taylor series backward, we have that 
	\begin{itemize}
		\item 	The zero-order Taylor-series approximation is
		$$f(x_{i-1}) \cong f(x_i) $$
		\item The first-order Taylor-series approximation is
		$$f(x_{i-1}) \cong f(x_i) -\frac{h}{1!}f'(x_i) $$
		\item The second-order Taylor series approximation is 
		$$f(x_{i-1}) \cong f(x_i) -\frac{h}{1!}f'(x_i) + \frac{h^2}{2!}f''(x_i)$$
		\item The third-order Taylor series approximate can be written as
		$$f(x_{i-1}) \cong f(x_i) - \frac{h}{1!}f'(x_i) + \frac{h^2}{2!}f''(x_i)-\frac{h^3}{3!}f^{(3)}(x_i)+...$$
	\end{itemize}	
	
\end{frame}

%%%%%%%%%%%%%%%%%%%%%%%%%%%%%%%%%%%%%%%%%%%%%%%%%%%%%%%%%%%%%%%%%%%%%%%
\begin{frame}[t]
	\frametitle{Approximating Derivatives - Backward difference}
	If we again wish to approximate the first derivative, we rearrange the first-order Taylor series approximation
	$$f(x_{i-1}) \cong f(x_i)- \frac{h}{1!}f'(x_i),$$
	to solve for the derivative, $f'(x_i)$
	$$f'(x_i) = \frac{f(x_{i})-f(x_{i-1})}{h} $$
\end{frame}

\begin{frame}[t]
	\frametitle{Approximating Derivatives - Backward difference}
	If we again wish to approximate the first derivative, we rearrange the first-order Taylor series approximation
	$$f(x_{i-1}) \cong f(x_i)- \frac{h}{1!}f'(x_i),$$
	to solve for the derivative, $f'(x_i)$
	$$f'(x_i) = \frac{f(x_{i})-f(x_{i-1})}{h} $$
	If we have equispaced data, we can think of $x_{i+1}$ as being the x located distance $h$ away from $x_i$. We can rewrite out formula as
	$$f'(x) = \frac{f(x)-f(x-h)}{h}  $$
\end{frame}

\begin{frame}[t]
	\frametitle{Approximating Derivatives - Backward difference}
	If we again wish to approximate the first derivative, we rearrange the first-order Taylor series approximation
	$$f(x_{i-1}) \cong f(x_i)- \frac{h}{1!}f'(x_i),$$
	to solve for the derivative, $f'(x_i)$
	$$f'(x_i) = \frac{f(x_{i})-f(x_{i-1})}{h} $$
	If we have equispaced data, we can think of $x_{i+1}$ as being the x located distance $h$ away from $x_i$. We can rewrite out formula as
	$$f'(x) = \frac{f(x)-f(x-h)}{h}  $$
	\\\vspace{10pt}
	This is the \textit{backward difference} scheme and it's approximation is $\mathcal{O}(h)$ accurate.
\end{frame}



%%%%%%%%%%%%%%%%%%%%%%%%%%%%%%%%%%%%%%%%%%%%%%%%%%%%%%%%%%%%%%%%%%%%%%%
\begin{frame}[t]
	\frametitle{Example Forward and Backward difference: $h= 0.5$}
	Use forward and backward difference approximations to estimate the first derivative of
	$$f(x) = -0.1x^4-0.15x^3-0.5x^2-0.25x+1.2 $$
	at $x=0.5$, using a step size $h = 0.5$. Note that the derivative of this function is
	$$f'(x) = -0.4x^3-0.45x^2-x-0.25, \qquad f'(0.5) = -0.9125$$
\end{frame}

\begin{frame}[t]
	\frametitle{Example Forward and Backward difference: $h= 0.5$}
	Use forward and backward difference approximations to estimate the first derivative of
	$$f(x) = -0.1x^4-0.15x^3-0.5x^2-0.25x+1.2 $$
	at $x=0.5$, using a step size $h = 0.5$. Note that the derivative of this function is
	$$f'(x) = -0.4x^3-0.45x^2-x-0.25, \qquad f'(0.5) = -0.9125$$
	\\\vspace{5pt}
	Let's start with the forward difference scheme.\\\vspace{5pt}
	We have
	\begin{align*}
		x = 0.5, \qquad &f(x) = 0.925\\
		x+h = 1, \qquad &f(x+h) = 0.2\\
	\end{align*}
\end{frame}

\begin{frame}[t]
	\frametitle{Example Forward and Backward difference: $h= 0.5$}
	Use forward and backward difference approximations to estimate the first derivative of
	$$f(x) = -0.1x^4-0.15x^3-0.5x^2-0.25x+1.2 $$
	at $x=0.5$, using a step size $h = 0.5$. Note that the derivative of this function is
	$$f'(x) = -0.4x^3-0.45x^2-x-0.25, \qquad f'(0.5) = -0.9125$$
	\\\vspace{5pt}
	Let's start with the forward difference scheme.\\\vspace{5pt}
	We have
	\begin{align*}
		x = 0.5, \qquad &f(x) = 0.925\\
		x+h = 1, \qquad &f(x+h) = 0.2\\
	\end{align*}
	Plugging this into our forward difference scheme, 
	$$f'(x) = \frac{f(x+h)-f(x)}{h} $$
	we have
	\begin{align*}
		f'(0.5) &\approx \frac{0.2-0.925}{0.5} \\
		&\approx -1.45
	\end{align*}
\end{frame}

\begin{frame}
	\frametitle{Example Forward and Backward difference: $h= 0.5$}
	Use forward and backward difference approximations to estimate the first derivative of
	$$f(x) = -0.1x^4-0.15x^3-0.5x^2-0.25x+1.2 $$
	at $x=0.5$, using a step size $h = 0.5$. Note that the derivative of this function is
	$$f'(x) = -0.4x^3-0.45x^2-x-0.25, \qquad f'(0.5) = -0.9125$$
	\\\vspace{5pt}
	Let's start with the forward difference scheme.\\\vspace{5pt}
	We have
	\begin{align*}
		x = 0.5, \qquad &f(x) = 0.925\\
		x+h = 1, \qquad &f(x+h) = 0.2\\
	\end{align*}
	Plugging this into our forward difference scheme, 
	$$f'(x) = \frac{f(x+h)-f(x)}{h} $$
	we have
	\begin{align*}
		f'(0.5) &\approx \frac{0.2-0.925}{0.5} \\
		&\approx -1.45
	\end{align*}
	The percent relative error is $58.9\%$.
\end{frame}

%%%%%%%%%%%%%%%%%%%%%%%%%%%%%%%%%%%%%%%%%%%%%%%%%%%%%%%%%%%%%%%%%%%%%%%

\begin{frame}
	\frametitle{Example Forward and Backward difference: $h= 0.5$}
	Use forward and backward difference approximations to estimate the first derivative of
	$$f(x) = -0.1x^4-0.15x^3-0.5x^2-0.25x+1.2 $$
	at $x=0.5$, using a step size $h = 0.5$. Note that the derivative of this function is
	$$f'(x) = -0.4x^3-0.45x^2-x-0.25, \qquad f'(0.5) = -0.9125$$
	\\\vspace{5pt}
	Now, applying the backward difference scheme.\\\vspace{5pt}
	We have
	\begin{align*}
		x = 0.5, \qquad &f(x) = 0.925\\
		x-h = 0, \qquad &f(x-h) = 1.2\\
	\end{align*}
\end{frame}


\begin{frame}
	\frametitle{Example Forward and Backward difference: $h= 0.5$}
	Use forward and backward difference approximations to estimate the first derivative of
	$$f(x) = -0.1x^4-0.15x^3-0.5x^2-0.25x+1.2 $$
	at $x=0.5$, using a step size $h = 0.5$. Note that the derivative of this function is
	$$f'(x) = -0.4x^3-0.45x^2-x-0.25, \qquad f'(0.5) = -0.9125$$
	\\\vspace{5pt}
	Now, applying the backward difference scheme.\\\vspace{5pt}
	We have
	\begin{align*}
		x = 0.5, \qquad &f(x) = 0.925\\
		x-h = 0, \qquad &f(x-h) = 1.2\\
	\end{align*}
	Plugging this into our backward difference scheme, 
	$$f'(x) = \frac{f(x)-f(x-h)}{h} $$
	we have
	\begin{align*}
		f'(0.5) &\approx \frac{0.925-1.2}{0.5} \\
		&\approx -0.55
	\end{align*}
\end{frame}


\begin{frame}
	\frametitle{Example Forward and Backward difference: $h= 0.5$}
	Use forward and backward difference approximations to estimate the first derivative of
	$$f(x) = -0.1x^4-0.15x^3-0.5x^2-0.25x+1.2 $$
	at $x=0.5$, using a step size $h = 0.5$. Note that the derivative of this function is
	$$f'(x) = -0.4x^3-0.45x^2-x-0.25, \qquad f'(0.5) = -0.9125$$
	\\\vspace{5pt}
	Now, applying the backward difference scheme.\\\vspace{5pt}
	We have
	\begin{align*}
		x = 0.5, \qquad &f(x) = 0.925\\
		x-h = 0, \qquad &f(x-h) = 1.2\\
	\end{align*}
	Plugging this into our backward difference scheme, 
	$$f'(x) = \frac{f(x)-f(x-h)}{h} $$
	we have
	\begin{align*}
		f'(0.5) &\approx \frac{0.925-1.2}{0.5} \\
		&\approx -0.55
	\end{align*}
	The percent relative error is $39.7\%$.
\end{frame}

%%%%%%%%%%%%%%%%%%%%%%%%%%%%%%%%%%%%%%%%%%%%%%%%%%%%%%%%%%%%%%%%%%%%%%%

\begin{frame}
	\frametitle{Example Forward and Backward difference: $h= 0.25$}
	Use forward and backward difference approximations to estimate the first derivative of
	$$f(x) = -0.1x^4-0.15x^3-0.5x^2-0.25x+1.2 $$
	at $x=0.5$, using a step size $h = 0.25$. Note that the derivative of this function is
	$$f'(x) = -0.4x^3-0.45x^2-x-0.25, \qquad f'(0.5) = -0.9125$$
	\\\vspace{5pt}
	Starting again with the forward difference scheme.\\\vspace{5pt}
	We have
	\begin{align*}
		x = 0.5, \qquad &f(x) = 0.925\\
		x+h = 0.75, \qquad &f(x+h) = 0.63632813\\
	\end{align*}
\end{frame}


\begin{frame}
	\frametitle{Example Forward and Backward difference: $h= 0.25$}
	Use forward and backward difference approximations to estimate the first derivative of
	$$f(x) = -0.1x^4-0.15x^3-0.5x^2-0.25x+1.2 $$
	at $x=0.5$, using a step size $h = 0.25$. Note that the derivative of this function is
	$$f'(x) = -0.4x^3-0.45x^2-x-0.25, \qquad f'(0.5) = -0.9125$$
	\\\vspace{5pt}
	Starting again with the forward difference scheme.\\\vspace{5pt}
	We have
	\begin{align*}
		x = 0.5, \qquad &f(x) = 0.925\\
		x+h = 0.75, \qquad &f(x+h) = 0.63632813\\
	\end{align*}
	Plugging this into our forward difference scheme, 
	$$f'(x) = \frac{f(x+h)-f(x)}{h} $$
	we have
	\begin{align*}
		f'(0.5) &\approx \frac{0.63632813-0.925}{0.5} \\
		&\approx -1.155
	\end{align*}
\end{frame}


\begin{frame}
	\frametitle{Example Forward and Backward difference: $h= 0.25$}
	Use forward and backward difference approximations to estimate the first derivative of
	$$f(x) = -0.1x^4-0.15x^3-0.5x^2-0.25x+1.2 $$
	at $x=0.5$, using a step size $h = 0.25$. Note that the derivative of this function is
	$$f'(x) = -0.4x^3-0.45x^2-x-0.25, \qquad f'(0.5) = -0.9125$$
	\\\vspace{5pt}
	Starting again with the forward difference scheme.\\\vspace{5pt}
	We have
	\begin{align*}
		x = 0.5, \qquad &f(x) = 0.925\\
		x+h = 0.75, \qquad &f(x+h) = 0.63632813\\
	\end{align*}
	Plugging this into our forward difference scheme, 
	$$f'(x) = \frac{f(x+h)-f(x)}{h} $$
	we have
	\begin{align*}
		f'(0.5) &\approx \frac{0.63632813-0.925}{0.5} \\
		&\approx -1.155
	\end{align*}
	The percent relative error is $26.5\%$. Roughly half of what it was when $h=0.5$!
\end{frame}

%%%%%%%%%%%%%%%%%%%%%%%%%%%%%%%%%%%%%%%%%%%%%%%%%%%%%%%%%%%%%%%%%%%%%%%

\begin{frame}
	\frametitle{Example Forward and Backward difference: $h= 0.25$}
	Use forward and backward difference approximations to estimate the first derivative of
	$$f(x) = -0.1x^4-0.15x^3-0.5x^2-0.25x+1.2 $$
	at $x=0.5$, using a step size $h = 0.5$. Note that the derivative of this function is
	$$f'(x) = -0.4x^3-0.45x^2-x-0.25, \qquad f'(0.5) = -0.9125$$
	\\\vspace{5pt}
	Now, applying the backward difference scheme.\\\vspace{5pt}
	We have
	\begin{align*}
		x = 0.5, \qquad &f(x) = 0.925\\
		x-h = 0.25, \qquad &f(x-h) = 1.10351563\\
	\end{align*}
\end{frame}


\begin{frame}
	\frametitle{Example Forward and Backward difference: $h= 0.25$}
	Use forward and backward difference approximations to estimate the first derivative of
	$$f(x) = -0.1x^4-0.15x^3-0.5x^2-0.25x+1.2 $$
	at $x=0.5$, using a step size $h = 0.5$. Note that the derivative of this function is
	$$f'(x) = -0.4x^3-0.45x^2-x-0.25, \qquad f'(0.5) = -0.9125$$
	\\\vspace{5pt}
	Now, applying the backward difference scheme.\\\vspace{5pt}
	We have
	\begin{align*}
		x = 0.5, \qquad &f(x) = 0.925\\
		x-h = 0.25, \qquad &f(x-h) = 1.10351563\\
	\end{align*}
	Plugging this into our backward difference scheme, 
	$$f'(x) = \frac{f(x)-f(x-h)}{h} $$
	we have
	\begin{align*}
		f'(0.5) &\approx \frac{0.925-1.10351563}{0.5} \\
		&\approx -0.714
	\end{align*}
\end{frame}


\begin{frame}
	\frametitle{Example Forward and Backward difference: $h= 0.25$}
	Use forward and backward difference approximations to estimate the first derivative of
	$$f(x) = -0.1x^4-0.15x^3-0.5x^2-0.25x+1.2 $$
	at $x=0.5$, using a step size $h = 0.5$. Note that the derivative of this function is
	$$f'(x) = -0.4x^3-0.45x^2-x-0.25, \qquad f'(0.5) = -0.9125$$
	\\\vspace{5pt}
	Now, applying the backward difference scheme.\\\vspace{5pt}
	We have
	\begin{align*}
		x = 0.5, \qquad &f(x) = 0.925\\
		x-h = 0.25, \qquad &f(x-h) = 1.10351563\\
	\end{align*}
	Plugging this into our backward difference scheme, 
	$$f'(x) = \frac{f(x)-f(x-h)}{h} $$
	we have
	\begin{align*}
		f'(0.5) &\approx \frac{0.925-1.10351563}{0.5} \\
		&\approx -0.714
	\end{align*}
	The percent relative error is $21.1\%$, roughly half of the error when $h=0.5$.
\end{frame}

%%%%%%%%%%%%%%%%%%%%%%%%%%%%%%%%%%%%%%%%%%%%%%%%%%%%%%%%%%%%%%%%%%%%%%%


\begin{frame}
	\frametitle{Example Forward and Backward difference summary}
	Since both the forward and backward schemes are $\mathcal{O}(h)$ accurate, when we halve $h$, we expect to halve the error - which is what we observed in this example. 
\end{frame}

%%%%%%%%%%%%%%%%%%%%%%%%%%%%%%%%%%%%%%%%%%%%%%%%%%%%%%%%%%%%%%%%%%%%%%%
\section{Deriving the Centered Difference Scheme}
\begin{frame}
	\frametitle{Approximating Derivatives - Centered Difference}
	There's a third way to approximate the first derivative using Taylor series. 
	\\\vspace{5pt}
	If we subtract the third-order backward Taylor series expansion 
	$$f(x_{i-1}) = f(x_i) -\frac{h}{1!}f'(x_i) + \frac{h^2}{2!}f''(x_i)-\frac{h^3}{3!}f^{(3)}(x_i)$$
	from the third-order forward Taylor series expansion
	$$f(x_{i+1}) = f(x_i) +\frac{h}{1!}f'(x_i) + \frac{h^2}{2!}f''(x_i)+ \frac{h^3}{3!}f^{(3)}(x_i)$$
\end{frame}

%%%%%%%%%%%%%%%%%%%%%%%%%%%%%%%%%%%%%%%%%%%%%%%%%%%%%%%%%%%%%%%%%%%%%%%
\begin{frame}
	\frametitle{Approximating Derivatives - Centered Difference}
	We have
	\begin{align*}
		f(x_{i+1}) - f(x_{i-1}) &= f(x_i) +\frac{h}{1!}f'(x_i) + \frac{h^2}{2!}f''(x_i) + \frac{h^3}{3!}f^{(3)}(x_i)\\
		&  \qquad \qquad \qquad-\left(f(x_i) -\frac{h}{1!}f'(x_i) + \frac{h^2}{2!}f''(x_i)-\frac{h^3}{3!}f^{(3)}(x_i)\right)\\
		f(x_{i+1}) - f(x_{i-1}) &= 2hf'(x_i)+\frac{2h^3}{3!}f^{(3)}(x_i)\\
	\end{align*}
	Rearranging for $f'(x_i)$
	$$f'(x_i) = \frac{f(x_{i+1})-f(x_{i-1})}{2h} -\frac{h^2}{6}f^{(3)}(x_i)$$
	which can be re-written as
	$$f'(x) = \frac{f(x+h)-f(x-h)}{2h} -\mathcal{O}(h^2)$$
	This is the \textit{centered difference} scheme and has an accuracy of $\mathcal{O}(h^2)$.
\end{frame}
%%%%%%%%%%%%%%%%%%%%%%%%%%%%%%%%%%%%%%%%%%%%%%%%%%%%%%%%%%%%%%%%%%%%%%%
\begin{frame}
	\frametitle{Example Centered difference: $h= 0.5$}
	Use a centered difference approximation to estimate the first derivative of
	$$f(x) = -0.1x^4-0.15x^3-0.5x^2-0.25x+1.2 $$
	at $x=0.5$, using a step size $h = 0.5$. Note that the derivative of this function is
	$$f'(x) = -0.4x^3-0.45x^2-x-0.25, \qquad f'(0.5) = -0.9125$$
	\\\vspace{5pt}
	Now, applying the centered difference scheme.\\\vspace{5pt}
	We have
	\begin{align*}
		x+h = 1, \qquad &f(x+h) = 0.2\\
		x-h = 0, \qquad &f(x-h) = 1.2\\
	\end{align*}
\end{frame}

%%%%%%%%%%%%%%%%%%%%%%%%%%%%%%%%%%%%%%%%%%%%%%%%%%%%%%%%%%%%%%%%%%%%%%%
\begin{frame}
	\frametitle{Example Centered difference: $h= 0.5$}
	Use a centered difference approximation to estimate the first derivative of
	$$f(x) = -0.1x^4-0.15x^3-0.5x^2-0.25x+1.2 $$
	at $x=0.5$, using a step size $h = 0.5$. Note that the derivative of this function is
	$$f'(x) = -0.4x^3-0.45x^2-x-0.25, \qquad f'(0.5) = -0.9125$$
	\\\vspace{5pt}
	Now, applying the centered difference scheme.\\\vspace{5pt}
	We have
	\begin{align*}
		x+h = 1, \qquad &f(x+h) = 0.2\\
		x-h = 0, \qquad &f(x-h) = 1.2\\
	\end{align*}
	Plugging this into our centered difference scheme, 
	$$f'(x) = \frac{f(x+h)-f(x-h)}{2h} $$
	we have
	\begin{align*}
		f'(0.5) &\approx \frac{0.2-1.2}{1} \\
		&\approx -1
	\end{align*}
	The percent relative error is $9.6\%$.
\end{frame}

%%%%%%%%%%%%%%%%%%%%%%%%%%%%%%%%%%%%%%%%%%%%%%%%%%%%%%%%%%%%%%%%%%%%%%%
\begin{frame}
	\frametitle{Example Centered difference: $h= 0.25$}
	Use a centered difference approximation to estimate the first derivative of
	$$f(x) = -0.1x^4-0.15x^3-0.5x^2-0.25x+1.2 $$
	at $x=0.5$, using a step size $h = 0.25$. Note that the derivative of this function is
	$$f'(x) = -0.4x^3-0.45x^2-x-0.25, \qquad f'(0.5) = -0.9125$$
	\\\vspace{5pt}
	Now, applying the centered difference scheme.\\\vspace{5pt}
	We have
	\begin{align*}
		x+h = 0.75, \qquad &f(x+h) = 0.63632813\\
		x-h = 0.25, \qquad &f(x-h) = 1.10351563\\
	\end{align*}
	Plugging this into our centered difference scheme, 
	$$f'(x) = \frac{f(x+h)-f(x-h)}{2h} $$
	we have
	\begin{align*}
		f'(0.5) &\approx \frac{0.63632813-1.10351563}{0.5} \\
		&\approx -0.934
	\end{align*}
	The percent relative error is $2.4\%$, a quarter of what is was when $h=0.5$.
\end{frame}

\begin{frame}
	\frametitle{Example centered difference summary}
	Since the centered difference scheme is $\mathcal{O}(h^2)$ accurate, when we halve $h$, we expect to quarter the error - which is what we observed in this example.
	$$\left(\frac{h}{2}\right)^2 = \frac{h^2}{4} $$ 
\end{frame}

%%%%%%%%%%%%%%%%%%%%%%%%%%%%%%%%%%%%%%%%%%%%%%%%%%%%%%%%%%%%%%%%%%%%%%%
\section{Programming Finite Difference Methods}
\begin{frame}
	\frametitle{Let's code them!}
	We'll write 3 routines\\
	\texttt{ }\\
	\texttt{forwardDifference(double x, double h)}\\
	\texttt{\qquad return (f(x + h) - f(x)) / h}\\
	\texttt{ }\\
	\texttt{backwardDifference(double x, double h)}\\
	\texttt{\qquad return (f(x) - f(x - h)) / h}\\
	\texttt{ }\\
	\texttt{centeredDifference(double x, double h)}\\
	\texttt{\qquad return (f(x + h) - f(x - h)) / (2 * h)}\\
	\texttt{ }\\
	And test it on the example
	$$f(x) = -0.1x^4-0.15x^3-0.5x^2-0.25x+1.2 $$
\end{frame}

\begin{frame}
	\frametitle{Activity}
	Add these functions to your library using the following syntax:\\
	\texttt{ }\\
	\texttt{forwardDifference(double x, double h, double (*f)(double x)}\\
	\texttt{ }\\
	\texttt{backwardDifference(double x, double h, double (*f)(double x))}\\
	\texttt{ }\\
	\texttt{centeredDifference(double x, double h, double (*f)(double x))}\\
	\texttt{ }\\
\end{frame}


\end{document}

\documentclass[11pt]{article}
\usepackage{amsmath,amssymb,amsthm}
\usepackage{tabu}
\usepackage{graphicx}
%\usepackage[bw]{mcode}
\usepackage[margin=.6in]{geometry}
\usepackage{tikz}
\usepackage{float}
\usepackage{textcomp}
\usepackage{multicol}
\addtolength{\topmargin}{.5in}
\usepackage{fancyhdr}
\usetikzlibrary{positioning}
\usepackage{pgfplots}
\setlength{\parindent}{0pt}
\setlength{\parskip}{5pt plus 1pt}
\setlength{\headheight}{20pt}
\renewcommand{\headrulewidth}{0pt}
\setlength{\headheight}{30.0pt}
\newcommand\question[2]{\vspace{.25in}\hrule\textbf{#1: #2}\vspace{.5em}\hrule\vspace{.10in}}
\renewcommand\part[1]{\vspace{.10in}\textbf{(#1)}}
\newcommand\enter{\vspace{.50in}}
\newcommand\algorithm{\vspace{.10in}\textbf{Algorithm:}}
\newcommand\correctness{\vspace{.10in}\textbf{Correctness: }}
\newcommand\runtime{\vspace{.10in}\textbf{Running time: }}
\pagestyle{fancyplain}

\begin{document}\raggedright
\newcommand\Page{\page  / \lastPage}
\newcommand\page{1}
\newcommand\qN[2]{\Large {#1} \small{#2} \normalsize}

% info
\newcommand\dueDate{\today}
\newcommand\hwnum{4}
\newcommand\ExNum{}

\newcommand\lastPage{3}

% set info
\lhead{\large Homework \hwnum }
\rhead{\rightHead}
\chead{\LARGE{Adding \texttt{NewtonRaphson} to your Library}}
\newcommand\rightHead{\large Due Oct 22, 2021}

\question{1}{Newton-Raphson method ( pts)}
Add your Newton-Raphson function to your library. Things to keep in mind:
\begin{itemize}
	\item Name Your function \textbf{NewtonRaphson}
	\item Use this order for your parameter list:
	\begin{enumerate}
		\item initial guess, \textbf{x}
		\item tolerance, \textbf{tol}
		\item maximum iterations, \textbf{Maxit}
		\item function, \textbf{f}
		\item function derivative, \textbf{df} 
	\end{enumerate}
	\item The Newton-Raphson declaration belongs in the RootFinding class with Bisect.
	\item The Newton-Raphson code belongs in the same .cpp file as the Bisect code.  
	\item Call NewtonRaphson the same way you called Bisect in the \texttt{main()}
	\item You will need to pass in pointers to the functions f and df. Define df in the same manner as f in the .cpp file containing the \texttt{main()}.
\end{itemize}

\vspace{1cm}

\question{2}{Test your library ( pts)}
Test your Newton-Raphson code on the function covered in slide 4 of lecture 12. Report the value of the root. \vspace{1cm}

\question{Submission Details}{}
Submit a \textbf{.zip} file of your visual studio project to Canvas.\\\vspace{0.5cm}

\end{document}
